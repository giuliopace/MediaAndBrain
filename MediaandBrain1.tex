\documentclass[conference]{IEEEtran}
\IEEEoverridecommandlockouts
% The preceding line is only needed to identify funding in the first footnote. If that is unneeded, please comment it out.
\usepackage{cite}
\usepackage{amsmath,amssymb,amsfonts}
\usepackage{algorithmic}
\usepackage{graphicx}
\usepackage{textcomp}
\usepackage{xcolor}
\def\BibTeX{{\rm B\kern-.05em{\sc i\kern-.025em b}\kern-.08em
    T\kern-.1667em\lower.7ex\hbox{E}\kern-.125emX}}
\begin{document}
\bibliographystyle{plain}

\title{Paper Title*\\
{\footnotesize \textsuperscript{*}Note: Sub-titles are not captured in Xplore and
should not be used}
\thanks{Identify applicable funding agency here. If none, delete this.}
}

\author{\IEEEauthorblockN{1\textsuperscript{st} Given Name Surname}
\IEEEauthorblockA{\textit{dept. name of organization (of Aff.)} \\
\textit{name of organization (of Aff.)}\\
City, Country \\
email address}
\and
\IEEEauthorblockN{2\textsuperscript{nd} Given Name Surname}
\IEEEauthorblockA{\textit{dept. name of organization (of Aff.)} \\
\textit{name of organization (of Aff.)}\\
City, Country \\
email address}
\and
\IEEEauthorblockN{3\textsuperscript{rd} Given Name Surname}
\IEEEauthorblockA{\textit{dept. name of organization (of Aff.)} \\
\textit{name of organization (of Aff.)}\\
City, Country \\
email address}
\and
\IEEEauthorblockN{4\textsuperscript{th} Given Name Surname}
\IEEEauthorblockA{\textit{dept. name of organization (of Aff.)} \\
\textit{name of organization (of Aff.)}\\
City, Country \\
email address}
\and
\IEEEauthorblockN{5\textsuperscript{th} Given Name Surname}
\IEEEauthorblockA{\textit{dept. name of organization (of Aff.)} \\
\textit{name of organization (of Aff.)}\\
City, Country \\
email address}
\and
\IEEEauthorblockN{6\textsuperscript{th} Given Name Surname}
\IEEEauthorblockA{\textit{dept. name of organization (of Aff.)} \\
\textit{name of organization (of Aff.)}\\
City, Country \\
email address}
}

\maketitle

\begin{abstract}
	This document is a model and instructions for \LaTeX.
	This and the IEEEtran.cls file define the components of your paper [title, text, heads, etc.]. *CRITICAL: Do Not Use Symbols, Special Characters, Footnotes,
	or Math in Paper Title or Abstract.
\end{abstract}

\begin{IEEEkeywords}
	component, formatting, style, styling, insert
\end{IEEEkeywords}

\section{Introduction}
	INSERT INTRODUCTION HERE

\section{Adaptation and Learning in the psychological literature}

	In this chapter we will focus on comparing different definitions of the terms “learning” and “adaptation” within existing literature and papers in the psychological science field. The goal is to create an overview of possible definitions by looking at similarities as well as differences.

	At first, we will take a look at the term “adaptation”. In their publication “Evaluating Evidence of Psychological Adaptation: How Do We Know One When We See One?” \cite{b1}, David P. Schmitt and June J. Pilcher refer to adaptation as both, a verb and a noun:
	\begin{itemize}
		\item Adaptation as a verb defines the process of evolution, where an animal (human or non-human) will change either on an individual basis or over the lifespan of its species to overcome challenges in its environment and become better suited to survive in the given environment;
		\item Adaptation as a noun defines the outcome or product of evolution, where they further narrow it down to two basic forms. On the one side, adaptation might be seen as a characteristic feature of an animal in order to survive or reproduce in its given environment, on the other hand, it can also be defined as the “historical end products of the process of evolution”.
	\end{itemize}

	The term “learning” can be seen similarly to the term “adaptation”. G. Hall defines the act of learning in \cite{b2} as the process by which a creature “interacts with its environment and becomes changed by this experience so that its subsequent behaviour is modified”. In \cite{b3}, learning is also described as a process of gaining new characteristic features based on study or experience. Linking the process of adapting to changes and learning based on experience, Robert A. Rescorla defines the term “learning” in \cite{b4} as “a process by which an organism benefits from experience so that its future behaviour is better adapted to its environment”.

	Another way of looking at the process of learning in a psychology is to describe it as a change in a creature’s behaviour, similarly to what adaptation as a noun can be seen as. Jan De Houwer describes learning as a set of changes in a creature’s behaviour resulting from patterns in its environment \cite{b5}, resulting from his observations from two other definitions, functionally as “changes in behavior that result from experience” and mechanistically as “changes in the organism that result from experience”.

	In \cite{b6}, the definition of the term “learning” is also seen as problematic, as it differs within and across disciplines of research. The conducted study reviewed several scientific papers and publications and concluded in a set of classes: (1) learning defined as the processing of information or experience, (2) learning defined as behavioural change and (3) learning defined as changes in behavioural mechanisms.

\section{Adaptation and Learning in neurology}
	We refer at ``learning'' as the process of acquiring new knowledge or modifying the existing one. Knowledge can not only be about notions but also behaviors, skills, value or preferences.\cite{b7}
	``Memory'' is what allow us to store and retrieve the knowledge we gather through the learning process.
	In neurology it is hard to talk about learning without talking about memory and viceversa. In this chapter we will focus on the different kinds of memory and the brain areas responsible for them.

	\subsection{Memory distinction}\label{MD}
		Memory is a general word used for both Short Term Memory and Long Term Memory. We are now going to focus on the latter, because is where we find the mechanism of adaptation overall of learning.
		We distinguish the Long Term Memory in two categories: ``Explicit Memory'' and ``Implicit Memory''.
		The first kind, also referred as ``Declarative memory'',  stores information about facts, things, people and places. In order for the piece of information to be retrieved, we are required an effort.
		The Explicit Memory is further differentiated in two sub-categories: ``Semantic Memory'', that is about general knowledge of facts about the world around us, and Episodic Memory, in charge of storing information regarding a specific time and space.
		For example, the concept of bicycle is delegated to the Semantic Memory, while the Episodic Memory will take care of remembering episodes related to bicycles.
		The Implicit Memory, also called ``Nondeclarative Memory'', is about learning physical acts or other skills that are recalled automatically and without an active thought towards it.
		It is divided in ``Priming'', ``Procedural Memory'', ``Associative Memory'' and ``Non-Associative Memory''.
		\begin{itemize}
			\item Priming is the influence that a stimulus has on subsequent stimuli;
			\item Procedural Memory is the kind of memory used to learn how to perform specific tasks without thinking. Going back to the bicycle example, this kind of memory ensures that, once learned how to ride a bike, we will be able to ride it without thinking;
			\item The Associative Memory regards the creation of a relation between two or more stimuli or between a stimulus and an associated behavior, as proved by the experiments conducted by Pavlov and Skinner;
			\item The Non-Associative Memory manages the learning of the properties of a specific stimulus. There are two opposite kinds of non-associative learning: Habituation and sensitization:
			\begin{itemize}
				\item Habituation is defined as a decrease in response to a stimulus repeated in time. The stimulus is marked as ``not harmful'' or ``not useful'' so the subject will start to ignore it.
				\item Sensitization is the opposite phenomenon: after an intense or a noxious stimulus, the subject will overreact to similar stimuli in order to protect itself.
			\end{itemize}
		\end{itemize}

	\subsection{Adaptation}\label{Ad}
		Adaptation is a similar concept to Habituation, but if the latter occurs with stimuli that occur periodically, the first is observed when there is a constant and continuous stimulus.
		On the physical level, mammals and other living organisms are constantly monitoring their environment through receptor cells that allow them to receive and process the stimuli. Through a ion flow, positive or negative feedback are sent to the receptors to allow the cells to close or open channel according to the ion flow. \cite{b8}
		It is believed that adaptation could modify the response range of the neurons in order to encode the signals to a more suitable strength. \cite{b9}

	\subsection{Anatomy of the learning process}\label{ALP}
		Learning is a general term that includes two processes in it: ``Explicit Learning'' and ``Implicit Learning''.
		\begin{itemize}
			\item Explicit Learning refers to the of creating Explicit Memories and it occurs in the hippocampus, neo-cortex and amygdala. \cite{b10} The hippocampus is where episodic memories are created and stored and it is a part of the temporal lobe. After some time in the hippocampus, some memories are moved to the neo-cortex as general knowledge. The amygdala is a structure shaped like an almond that is in the temporal lobe, and gives an emotional background to the memory.
			\item Implicit Learning is the process of creating Implicit Memories and rely on basal ganglia and cerebellum. \cite{b11} Basal ganglia is what we use to store Procedural Memories, while the cerebellum is involved in several motor function like coordination.
		\end{itemize}

\section{Adaptation and Learning in other areas of science}
	Aside from psychology and neurology, which could be seen as research areas related to the human and non-human mind and behaviour, there are many different other areas of research where learning and adaptation is one of the major focus topics. In [6], several other research disciplines are listed, containing but not limited to (1) behavioural ecology, (2) evolutionary theory and (3) computer science. In this chapter, we will cover a selection of different research areas related to learning and adaptation.

	\subsection{Educational Studies}\label{ES}
	Education starts at the early age of an infant, where learning is a major focus in research, e.g. how infants are able to learn about the physical world they live in \cite{b12}. Another research conducted in the area of education is related to the social environment of students and the definition of a professional learning community \cite{b13}.

	\subsection{Linguistics}\label{Li}
	Within the field of linguistics, research on how to acquire the skill to speak a foreign language has emerged as a discipline related to the term learning \cite{b14}.

	\subsection{Computer science}\label{CS}
	WRITE HERE COMPUTER SCIENCE STUFF


\section{A model of conciousness for Adaptation and Learning}

\section{Adaptation and Learning in deep neural networks}

\section{Conclusion}
	INSERT CONCLUSION HERE

\section*{References}

\begin{thebibliography}{00}

%chapter 2
\bibitem{b1} D. P. Schmitt and J. J. Pilcher (2004), “Evaluating Evidence of Psychological Adaptation: How Do We Know One When We See One?”, in Psychological science, vol. 15, issue 10, pp. 643-649, October 2004.
\bibitem{b2} G. Hall (2003), “The Psychology of Learning”, in Encyclopedia of cognitive science, vol. 2, pp. 837-845.
\bibitem{b3} S. M. Breedlove and N. V. Watson (2013), “Biological Psychology: An Introduction to Behavioral, Cognitive, and Clinical Neuroscience”, 7th edition, Sinauer Associates Inc.
\bibitem{b4} R. A. Rescorla (1988), “Behavioral studies of Pavlovian conditioning”, in Ann. Rev. Neurosci., vol. 11, pp. 329-352.
\bibitem{b5} J. De Houwer (2013), “What is learning? On the nature and merits of a functional definition of learning.”, in Psychonomic Bulletin \& Review, vol. 20, issue 4, pp. 631-642.
\bibitem{b6} A. B. Barron, et al. (2015), “Embracing multiple definitions of learning”, in Trends of Neurosciences, vol. 38, issue 7, pp. 405-407.

%chapter 3
\bibitem{b7} R. Gross (2010), Psychology: The Science of Mind and Behaviour, 6th Edition, Hodder Education, chapter 1
\bibitem{b8} Dougherty, D. P.; Wright, G. A.; Yew, A. C. (2005). Computational model of the cAMP-mediated sensory response and calcium-dependent adaptation in vertebrate olfactory receptor neurons. Proceedings of the National Academy of Sciences. 102 (30): 10415–20.
\bibitem{b9} Chung, S; Li, X; Nelson, S. B. (2002). Short-term depression at thalamocortical synapses contributes to rapid adaptation of cortical sensory responses in vivo. Neuron. 34 (3): 437–46.
\bibitem{b10} Larry R. Squire. The Legacy of Patient H.M. for Neuroscience.
\bibitem{b11} Stocco, Andrea; Lebiere, Christian; Anderson, John R. (2010). Conditional Routing of Information to the Cortex: A Model of the Basal Ganglia's Role in Cognitive Coordination. Psychological Review. 117 (2): 541–74.

%chapter 4
\bibitem{b12} R. Baillargeon (1994), “How do infants learn about the physical world?”, in Current Directions in Psychological Science, vol. 3, issue 5, pp. 133-140.
\bibitem{b13} R. DuFour (2004), “What Is a ‘Professional Learning Community’?”, in Educational Leadership, vol. 61, issue 8, pp. 6-10.
\bibitem{b14} S. Gass, et al. (1989), “Linguistic Perspectives on Second Language Acquisition”, Cambridge University Press.

%chapter 5

%chapter 6

%chapter 7

%chapter 8

\end{thebibliography}
\vspace{12pt}

\end{document}
